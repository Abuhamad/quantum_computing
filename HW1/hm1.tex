\documentclass[12pt]{article}
\usepackage{amsmath, amssymb, amsthm,qcircuit}
\usepackage{comment}
\usepackage{hyperref}
\renewcommand{\>}{\rangle}
\newcommand{\<}{\langle}
\newcommand{\cL}{\mathcal{L}}
\newcommand{\cH}{\mathcal{H}}
\newcommand{\C}{\mathbb{C}}
\newcommand{\tr}{\mathrm{Tr}}

\setlength{\marginparwidth}{0pt} \setlength{\hoffset}{0cm}
\setlength{\oddsidemargin}{0pt} \setlength{\topmargin}{1cm}
\setlength{\headheight}{0pt} \setlength{\headsep}{0pt}
\addtolength{\textwidth}{3cm} \addtolength{\textheight}{5cm}

\begin{document}


\begin{center}
{\bf COT 5600 Quantum Computing} 

\medskip
{\bf Spring 2019}

\bigskip

{\bf Homework 1}
\end{center}

\newpage

%%%%%%%%%%%%%%%%%%%%%%%%%%%%%%%%%%%%%%%%% PROBLEM 1

\noindent {\bf Problem 1} (Eigenvalues of Pauli operators)

\medskip
\noindent
Let $B=\{|\psi_1\>,\ldots,|\psi_d\>\}$ and $B'=\{|\psi'_1\>,\ldots,|\psi'_d\>\}$ be two orthonormal bases of $\C^d$. 

The ONBs $B$ and $B'$ are called mutually unbiased if 
\[
|\< \psi_i | \psi'_j \>|^2 = \frac{1}{d} 
\]
for all $1\le i,j \le d$.

Show that the eigenbases of the Pauli operators 
\[
\sigma_x = 
\left(
\begin{array}{cc}
0 & 1 \\
1 & 0
\end{array}
\right),
%
\sigma_y = 
\left(
\begin{array}{cc}
0 & -i \\
i & 0
\end{array}
\right),
%
\sigma_z = 
\left(
\begin{array}{cc}
1 & 0 \\
0 & -1
\end{array}
\right)
\]
are mutually unbiased. Implement Python methods that compute the eigenbases of the Pauli operators and check that they form mutually unbiased bases. \newline




\textbf{Answer: } Each Pauli matrix has the eigenvalues 1 and −1, and their respective eigenvectors are as follows:
\begin{itemize}
    \item $\sigma_x = \left(\begin{array}{cc} 0 & 1 \\ 1 & 0 \end{array} \right)$ has the basis $|{\uparrow} \sigma_x \> = \frac{1}{\sqrt{2}} \left(\begin{array}{cc} 1 \\ 1 \end{array} \right)$ and $|{\downarrow} \sigma_x \> = \frac{1}{\sqrt{2}}\left(\begin{array}{cc} 1 \\ -1 \end{array} \right)$ 
    \item $\sigma_y = 
\left(
\begin{array}{cc}
0 & -i \\
i & 0
\end{array}
\right)$ has the basis $|{\uparrow} \sigma_y \> = \frac{1}{\sqrt{2}} \left(\begin{array}{cc} 1 \\ i \end{array} \right)$ and $|{\downarrow} \sigma_y \> = \frac{1}{\sqrt{2}}\left(\begin{array}{cc} 1 \\ -i \end{array} \right)$
    \item $\sigma_z = 
\left(
\begin{array}{cc}
1 & 0 \\
0 & -1
\end{array}
\right)$ has the basis $|{\uparrow} \sigma_z \> =  \left(\begin{array}{cc} 1 \\ 0 \end{array} \right)$ and $|{\downarrow} \sigma_z \> = \left(\begin{array}{cc} 0 \\ 1 \end{array} \right)$
\end{itemize}

To show that those eigenvectors are mutually unbiased, check the $|\< \psi_i | \psi'_j \>|^2 = \frac{1}{d}$, for example:

$$|\< \uparrow \sigma_z | \uparrow \sigma_x \>|^2 = | \left(\begin{array}{cc} 1 \quad 0 \end{array} \right)  \left(\begin{array}{cc} \frac{1}{\sqrt{2}} \\ \frac{1}{\sqrt{2}} \end{array} \right)|^2 = (\frac{1}{\sqrt{2}})^2 = \frac{1}{2}$$


\noindent
\textbf{Code:} Please, find the code in the following link.

\noindent
{\small \em \url{ https://github.com/Abuhamad/quantum_computing/blob/master/HW1/HW1_problem1.py}}


\newpage

%%%%%%%%%%%%%%%%%%%%%%%%%%%%%%%%%%%%%%%%% PROBLEM 2

\noindent {\bf Problem 2} (Trace inner product)

\medskip
\noindent
For $A,B\in\C^{d\times d}$, define
\[
\< A | B\>_{\mathrm{Tr}} = \mathrm{Tr}(A^\dagger B)\,.
\]
Prove that the above map defines an inner product on the vector space $\C^{d\times d}$. (In the literature, this inner product is called the trace inner product or Hilbert-Schmidt inner product.) \newline

\textbf{Answer: } Let's denote $\< . , . \>$ as the inner product instead of $\< . | . \>$ to avoid confusion between the inner product of matrices $\in \C^{d\times d}$ and vectors $\< bra | ket \>$.
To prove $\< A , B\>_{\mathrm{Tr}} = \mathrm{Tr}(A^\dagger B)$ is an inner product on $\C^{d\times d}$, the following properties should be fulfilled:
\begin{enumerate}
    \item \textbf{Conjugate-linearity in the first factor:}
    Consider $A, B, C \in\C^{d\times d}$, then:
    \begin{align*}
        \< A + B , C\> & = \mathrm{Tr}((A^\dagger + B^\dagger)  C) \\
                       & = \mathrm{Tr}(A^\dagger C + B^\dagger C) \\
                       & = \mathrm{Tr}(A^\dagger C) +  \mathrm{Tr}(B^\dagger C) \\
                       & = \<A, C\>_{\mathrm{Tr}} +  \<B, C\>_{\mathrm{Tr}}
    \end{align*}
    Moreover,
    \begin{align*}
        \< aA + bB , C\> & = \mathrm{Tr}((aA^\dagger + bB^\dagger)  C) \\
                       & = \mathrm{Tr}(aA^\dagger C + bB^\dagger C) \\
                       & = a \mathrm{Tr}(A^\dagger C) +  b \mathrm{Tr}(B^\dagger C) \\
                       & = a \<A, C\>_{\mathrm{Tr}} +  b \<B, C\>_{\mathrm{Tr}}
    \end{align*}
    \item \textbf{Linearity in the second factor:}
        \begin{align*}
        \< A , B + C\> & = \mathrm{Tr}(A^\dagger (B + C) \\
                       & = \mathrm{Tr}(A^\dagger B + A^\dagger C) \\
                       & =  \mathrm{Tr}(A^\dagger B) +   \mathrm{Tr}(A^\dagger C) \\
                       & =  \<A, B\>_{\mathrm{Tr}} +   \<A, C\>_{\mathrm{Tr}}
    \end{align*}
    Moreover,
    \begin{align*}
        \< A , bB\> & = \mathrm{Tr}(A^\dagger bB) \\
                       & = b~ \mathrm{Tr}(A^\dagger B) \\
                       & = b~ \<A, B\>_{\mathrm{Tr}}
    \end{align*}
    \item \textbf{Symmetry:} It is clear that $\< . , .\>$ is symmetric as $\mathrm{Tr}$ is transpose invariant.
    $$\< A, B \>_{\mathrm{Tr}} = \mathrm{Tr}(A^\dagger B) = \mathrm{Tr}((B A^\dagger)^\dagger) =  \< B, A \>_{\mathrm{Tr}}$$
    \item \textbf{Positivity:} For $A \in \C^{d\times d}$, the $\< A, A \>_{\mathrm{Tr}} = \mathrm{Tr}(A^\dagger A) = \sum_{i,j = 1}^{n} |a_{i,j}|^2$ which is always positive.
    
    
\end{enumerate}















\begin{comment}
First, the dot product $\< A | B\>$ is a scalar value (a matrix  $\in\C^{1\times 1}$), and it is calculated as 
\begin{equation}
\< A | B\> =  \sum_{i=1}^d a_i b_i 
\end{equation}

Second, the trace for a matrix is the summation of its diagonal elements. 

\begin{equation}
\mathrm{Tr} (A) =  \sum_{i=1}^d a_{ij}  \quad \forall i=j 
\end{equation}

Now, a {\em bra} vector $\< A| \in\C^{1\times d}$ dot {\em ket} vector $|B\> \in\C^{d\times 1}$ will result in  
$$
\begin{bmatrix} a_1 & a_2 & \cdots & a_d \end{bmatrix} \Dot{}
\begin{bmatrix} b_1 \\ b_2 \\ \vdots \\ b_d \end{bmatrix} 
$$
$$\< A | B\> =  \sum_{i=1}^d a_i b_i  \in\C^{1\times 1} = \< A | B\>_\mathrm{Tr}$$  


And the dot product of $A^\dagger \in \C^{d\times 1}$ and $B \in \C^{1\times d}$ will result in $(A^\dagger B) \in \C^{d\times d}$, which will be as the following: 

\[
    \begin{bmatrix} a_1 \\ a_2 \\ \vdots \\ a_d \end{bmatrix} \Dot{} \begin{bmatrix} b_1 \\ b_2 \\ \vdots \\ b_d \end{bmatrix} 
\]
\[
(A^\dagger B) = \begin{bmatrix} a_1 b_1 & a_1 b_2 & \cdots & a_1 b_d \\ a_2 b_1 & a_2 b_2 & \cdots & a_2 b_d \\ \vdots & \vdots & \vdots & \vdots \\ a_d b_1 & a_d b_2 & \cdots & a_d b_d 
\end{bmatrix}. 
\]

Notice the  $\mathrm{Tr}(A^\dagger B) = \sum_{i=1}^d a_i b_i = \< A | B\>_\mathrm{Tr}$
\end{comment}
 













\newpage

%%%%%%%%%%%%%%%%%%%%%%%%%%%%%%%%%%%%%%%%% PROBLEM 3

\noindent {\bf Problem 3} (Unitary error basis)

\noindent
Define the matrices $X,Z\in\C^{d\times d}$ as follows
\begin{eqnarray}
X & = & \sum_{k=0}^d |k+1\>\<k| \\
Z & = & \sum_{\ell=0}^{d-1} \omega^\ell |\ell\>\<\ell|
\end{eqnarray}
where the addition is modulo $k+1$ and $\omega=e^{2\pi i/d}$ is a primitive $d$th root of unity. Show that the $d^2$ matrices
\[
M^{(a,b)} = X^a Z^b
\]
where $a,b\in\{0,\ldots, d-1\}$ form an orthonormal basis with respect to the trace inner product. Implement methods in Python that construct these matrices and compute the trace inner product for all pairs. (The above collection of matrices is called a unitary error basis in the literature and is used, for instance, in the theory of quantum error correcting codes and quantum channels for qudit systems. It is a generalization of the Pauli basis for a qubit system to a qudit system.) \newline



\noindent
\textbf{Code:} Please, find the code in the following link.

\noindent
{\small \em \url{https://github.com/Abuhamad/quantum_computing/blob/master/HW1/HW1_problem3.py}}


\end{document}
